%===========================================================
% Semi-Processor ISA Reference — Minimal (Header + Bitfields)
%===========================================================
\documentclass[10pt]{article}

% Basic fonts (pdfLaTeX-friendly)
\usepackage[T1]{fontenc}
\usepackage[utf8]{inputenc}
\usepackage{lmodern}
\usepackage{amsmath}
\usepackage{amsmath}         % you already added this earlier
\usepackage{tabularx}
\usepackage{booktabs}
\usepackage{subcaption}      % provides subtable
\usepackage{array}
\newcolumntype{Y}{>{\raggedright\arraybackslash}X}

\usepackage{ltablex}   % longtable + tabularx hybrid
\keepXColumns
\usepackage{booktabs}
\usepackage{needspace} % keep headings with following material

% A4 portrait; we'll rotate only the bitfield page
\usepackage[a4paper,margin=1in]{geometry}
\usepackage{longtable} % multipage, non-floating tables
\usepackage{array}     % for custom p{..} column types
\newcolumntype{L}[1]{>{\raggedright\arraybackslash}p{#1}} % like 'X', but for longtable

% Diagrams
\usepackage{bytefield}
\usepackage{pdflscape}
% in preamble
\usepackage{graphicx} % (already common)
\usepackage{bytefield}
\usepackage{graphicx}
\usepackage{makecell} % for line breaks in narrow cells
\renewcommand\cellalign{cc}
\newcommand{\bfstack}[3]{% name, hi, lo
  \makecell{\strut \scriptsize #1\\ \tiny [#2:#3]}
}
\bytefieldsetup{
  bitwidth=5.0em,   % horizontal size of ONE bit
  bitheight=5.2ex,  % vertical size of ONE bit row
  boxformatting=\scriptsize % (optional) shrink text inside boxes
}
\usepackage{caption}
\usepackage{tabularx}
\usepackage{booktabs}   % nicer rules (optional)
\newcolumntype{Y}{>{\raggedright\arraybackslash}X} % auto-wrap ragged-right



% Small tick labels; set bitwidth to snap 128 bits to the line
\bytefieldsetup{bitformatting=\scriptsize}
\newcommand{\SetBitwidthToLine}{%
  \bytefieldsetup{bitwidth=\dimexpr\linewidth/128\relax}%
}

% Metadata
\newcommand{\archname}{LAPU-128}
\newcommand{\archver}{v0.6 (Draft)}
\newcommand{\EightBitTicks}{%
  \bitheader{0,8,16,24,32,40,48,56,64,72,80,88,96,104,112,120,127}\\%
}


\begin{document}

%======================
% Header (title) page
%======================
\begin{center}
  {\Large \textbf{\archname} \\ Instruction Set Reference}\\[4pt]
  {\small \archver\ — \today}
\end{center}

\begin{abstract}
This document outline the 128-bit instruction formats for LAPU-128,
focused on complex arithmetic and vector descriptors. LAPU-128 is small focused ISA designed perform complex tensor operations in an embedded enviroment. The following page
shows the canonical XL, XC, XV, and XM encodings with 8-bit tick marks.
\end{abstract}

\clearpage

%==========================================
% Landscape page: Bitfield diagrams only
%==========================================
\begin{landscape}
\section*{Core Instruction Formats (128-bit)}
\SetBitwidthToLine

% helper: shorter labels fit better
\newcommand{\slabel}[1]{\centering\scriptsize #1}

%================ R-Type ================
\subsection*{R-Type: Register-to-Register operations of either complex scalar or complex vector types}
\begin{bytefield}{128}
  \EightBitTicks
  \begin{rightwordgroup}{R}
    \bitbox{8}{\bfstack{opcode}{127}{120}} &
    \bitbox{8}{\bfstack{subop}{119}{112}} &
    \bitbox{16}{\bfstack{flags}{111}{96}} &
    \bitbox{3}{\bfstack{rd}{95}{93}} &
    \bitbox{3}{\bfstack{rs1}{92}{90}} &
    \bitbox{3}{\bfstack{rs2}{89}{87}} &
    \bitbox{16}{\bfstack{imm16}{86}{71}} &
    \bitbox{71}{\bfstack{reserved}{70}{0}}
  \end{rightwordgroup}
\end{bytefield}

\vspace{1.2em}

%================ I-Type ================
\subsection*{I-type: Immediate operations of just complex scalars}
\begin{bytefield}{128}
  \EightBitTicks
  \begin{rightwordgroup}{I}
    \bitbox{8}{\bfstack{opcode}{127}{120}} &
    \bitbox{8}{\bfstack{subop}{119}{112}} &
    \bitbox{16}{\bfstack{flags}{111}{96}} &
    \bitbox{3}{\bfstack{rd}{95}{93}} &
    \bitbox{3}{\bfstack{rs1}{92}{90}} &
    \bitbox{90}{\bfstack{imm\_90}{89}{0}}
  \end{rightwordgroup}
\end{bytefield}

\vspace{1.2em}

%================ J-Type ================
\subsection*{J (conditional jump, 128-bit descriptor)}
\begin{bytefield}{128}
  \EightBitTicks
  \begin{rightwordgroup}{JX}
    \bitbox{8}{\bfstack{opc}{127}{120}} &
    \bitbox{8}{\bfstack{subop}{119}{112}} &
    \bitbox{16}{\bfstack{flags16}{111}{96}} &
    \bitbox{3}{\bfstack{rs1}{95}{93}} &
    \bitbox{33}{\bfstack{offs33 (PC\_rel)}{92}{60}} &
    \bitbox{60}{\bfstack{reserved}{59}{0}}
  \end{rightwordgroup}
\end{bytefield}

\vspace{1.2em}

%================ S-Type ================
\subsection*{S-type: matrix-bank load/store (scalar \& vector), 128-bit}
\begin{bytefield}{128}
  \EightBitTicks
  \begin{rightwordgroup}{S-type (stride implicit)}
    \bitbox{8}{\bfstack{opc}{127}{120}} &
    \bitbox{8}{\bfstack{subop}{119}{112}} &
    \bitbox{16}{\bfstack{FLAGS16}{111}{96}} &
    \bitbox{3}{\bfstack{reg3}{95}{93}} &
    \bitbox{4}{\bfstack{mbid}{92}{89}} &
    \bitbox{16}{\bfstack{i16}{88}{73}} &
    \bitbox{16}{\bfstack{j16}{72}{57}} &
    \bitbox{16}{\bfstack{len16}{56}{41}} &
    \bitbox{41}{\bfstack{rsv}{40}{0}}
  \end{rightwordgroup}
\end{bytefield}

\end{landscape}

%======================
% Register Layout
%======================
\section*{Register Layout}

\subsection*{Architectural Registers (Summary)}
\begin{center}
\begin{table}[h]
\centering
\small
\begin{tabularx}{\linewidth}{|l|l|l|Y|}
\hline
\textbf{Class} & \textbf{Names} & \textbf{Width / Elements} & \textbf{Notes} \\
\hline
Scalar (complex) & $s0..s7$ & 128\,b each (complex Q32.32 + Q32.32) &
\textbf{$s0$ is hard-wired to 0}. $s1$ is the conventional branch predicate (0/1). \\
\hline
Vector (complex) & $v0..v7$ & VLEN elements; each element 128\,b complex &
\textbf{$v0$ is hard-wired to all-zeros}. Vector ops always operate on \textbf{all VLEN elements}. \\
\hline
\end{tabularx}
\end{table}

\end{center}

\subsection*{Vector Length}
\noindent \textbf{VLEN} is a hardware/HDL parameter fixed at synthesis time. It is constant at runtime. All vector instructions operate over the entire range $[0, \mathrm{VLEN}-1]$.

\subsection*{Complex Number Format (Q32.32 + Q32.32)}
Each scalar register and each vector element encodes a complex value $(\mathrm{Re}, \mathrm{Im})$ in fixed point:
\[
\mathrm{Re}, \mathrm{Im} \in \text{Q32.32 two’s complement} \quad\Rightarrow\quad
x_{\text{real}} = \frac{X_{\text{int}}}{2^{32}},\;\;
x_{\text{imag}} = \frac{Y_{\text{int}}}{2^{32}}.
\]
The 128-bit complex is stored little-endian in memory with \textbf{Re at the lower address} and \textbf{Im at the higher address}. Each half (Re or Im) is a 64-bit two’s-complement fixed-point integer with 32 integer bits and 32 fractional bits.

\subsection*{Endianness}
\noindent Instruction words (128\,b) and data are little-endian. For complex numbers in memory: bytes for \(\mathrm{Re}\) precede bytes for \(\mathrm{Im}\).

\subsection*{Zero Registers}
\noindent The following are architecturally fixed to zero and never written:
\[
s0 \equiv 0\quad\text{(complex zero)}, \qquad
v0[i] \equiv 0\;\; \forall\, i \in [0,\mathrm{VLEN}-1].
\]

%==============================
% Instruction Semantics section
%==============================
\clearpage
\section*{Instruction Semantics}

%---------------- R-type -----------------
\section*{R-type — Register-to-Register (complex)}
\subsection*{OPCODE: \texttt{0x01}}
\subsection*{Description}


These are register to register operations involve either two vectors, two scalars, or a vector and a scalar. To determine mapping check bit position [97:96] under flags. Additionally each subop code range will be defined per mapping\\\\

\[
\begin{aligned}
f(s_1,s_2)                 &\mapsto s \in \mathrm{S}            && \texttt{00} \\
f(\mathbf v_1,\mathbf v_2) &\mapsto \mathbf v \in \mathrm{V}    && \texttt{01} \\
f(\mathbf v,s)             &\mapsto \mathbf v \in \mathrm{V}    && \texttt{10} \\
f(\mathbf v,s)             &\mapsto s \in \mathrm{S}            && \texttt{11} \\
\end{aligned}
\]



The undefined flag fields are open to future use. 



%=== Scalar ops (unary and binary) in one table =====================
\subsection*{Scalar ops (unary and binary)}
\begin{longtable}{@{}l l l L{0.38\linewidth} L{0.30\linewidth}@{}}
\caption{Scalar register ops (\texttt{s*}): $\mathrm{S}\to\mathrm{S}$ and $\mathrm{S}\times\mathrm{S}\to\mathrm{S}$}\label{tab:scalar_ops_all}\\
\toprule
\textbf{Mnemonic} & \textbf{subop} & \textbf{Operands} & \textbf{Effect} & \textbf{Notes} \\
\midrule
\endfirsthead
\toprule
\textbf{Mnemonic} & \textbf{subop} & \textbf{Operands} & \textbf{Effect} & \textbf{Notes} \\
\midrule
\endhead

\multicolumn{5}{@{}l@{}}{\emph{Unary: $\mathrm{S}\to\mathrm{S}$}}\\
\addlinespace[2pt]
\texttt{cneg.c}    & 0x10 & d, a        & d $\leftarrow -a$                                                & Two’s-complement both halves. \\
\texttt{conj.c}    & 0x20 & d, a        & d $\leftarrow \mathrm{conj}(a)$                                  & Negate imaginary half. \\
\texttt{csqrt.c}   & 0x26 & d, a        & d $\leftarrow \sqrt{a}$                                          & Principal root; widen, then truncate to Q32.32+Q32.32. \\
\texttt{cabs2.c}   & 0x21 & d, a        & d$_\mathrm{re}\leftarrow \Re(a)^2{+}\Im(a)^2$, d$_\mathrm{im}\leftarrow 0$ & Magnitude$^2$; widen then truncate. \\
\texttt{cabs.c}    & 0x22 & d, a        & d$_\mathrm{re}\leftarrow \sqrt{\Re(a)^2+\Im(a)^2}$, d$_\mathrm{im}\leftarrow 0$ & Fixed-point $\sqrt{\cdot}$; truncating. \\
\texttt{creal.c}   & 0x50 & d, a        & d$_\mathrm{re}\leftarrow \Re(a)$,\; d$_\mathrm{im}\leftarrow 0$  & Extract real. \\
\texttt{cimag.c}   & 0x51 & d, a        & d$_\mathrm{re}\leftarrow \Im(a)$,\; d$_\mathrm{im}\leftarrow 0$  & Extract imaginary to real half. \\
\texttt{crecip.c}  & 0x25 & d, a        & d $\leftarrow 1 \div a$                                          & $(\overline{a})/|a|^2$; if $a{=}0$ then d$:=0$. \\
\addlinespace[4pt]
\cmidrule(lr){1-5}
\addlinespace[4pt]

\multicolumn{5}{@{}l@{}}{\emph{Binary: $\mathrm{S}\times\mathrm{S}\to\mathrm{S}$}}\\
\addlinespace[2pt]
\texttt{cadd.c}    & 0x00 & d, a, b & d $\leftarrow$ a $+$ b & Truncating Q32.32+Q32.32. \\
\texttt{csub.c}    & 0x01 & d, a, b & d $\leftarrow$ a $-$ b & Truncating. \\
\texttt{cmul.c}    & 0x02 & d, a, b & d $\leftarrow$ a $\times$ b & Widen internally, truncate to Q32.32+Q32.32. \\
\texttt{cdiv.c}    & 0x03 & d, a, b & d $\leftarrow$ a \,$\div$\, b & $(a\,\overline{b})/|b|^2$; if $|b|=0$ then d$:=0$. \\
\texttt{cmaxabs.c} & 0x62 & d, a, b & d $\leftarrow \arg\max_{x\in\{a,b\}} |x|$ & Ties pick \texttt{a}. \\
\texttt{cminabs.c} & 0x63 & d, a, b & d $\leftarrow \arg\min_{x\in\{a,b\}} |x|$ & Ties pick \texttt{a}. \\

\bottomrule
\end{longtable}



%=== V -> V (lane-wise) =============================================
\subsection*{Vector $\to$ Vector}
\begin{longtable}{@{}l l l L{0.38\linewidth} L{0.30\linewidth}@{}}
\caption{Lane-wise vector ops (\texttt{v*}): $\mathrm{V}\to\mathrm{V}$ and $\mathrm{V}\times\mathrm{V}\to\mathrm{V}$}\label{tab:v_to_v}\\
\toprule
\textbf{Mnemonic} & \textbf{subop} & \textbf{Operands} & \textbf{Effect} & \textbf{Notes} \\
\midrule
\endfirsthead
\toprule
\textbf{Mnemonic} & \textbf{subop} & \textbf{Operands} & \textbf{Effect} & \textbf{Notes} \\
\midrule
\endhead
\texttt{cadd.v} & 0x10 & vD, vA, vB & vD[i] $\leftarrow$ vA[i] $+$ vB[i]                  & Saturating per lane. \\
\texttt{csub.v} & 0x11 & vD, vA, vB & vD[i] $\leftarrow$ vA[i] $-$ vB[i]                  & Saturating per lane. \\
\texttt{cmul.v} & 0x12 & vD, vA, vB & vD[i] $\leftarrow$ vA[i] $\times$ vB[i]             & Complex lane-wise multiply. \\
\texttt{cmac.v} & 0x13 & vD, vA, vB & vD[i] $\leftarrow$ vD[i] + vA[i]\,$\times$\,vB[i]   & Fused complex MAC per lane. \\
\texttt{cdiv.v} & 0x16 & vD, vA, vB & vD[i] $\leftarrow$ vA[i] \,$\div$\, vB[i]           & $(a\,\overline{b})/|b|^2$; if $|b|{=}0$ then lane:=0. \\
\texttt{conj.v} & 0x21 & vD, vA     & vD[i] $\leftarrow$ \mathrm{conj}(vA[i])             & Lane-wise conjugate. \\
\bottomrule
\end{longtable}

%=== V / V,V -> S (reductions) =====================================
\subsection*{Vector / Vector $\to$ Scalar (reductions)}
\begin{longtable}{@{}l l l L{0.38\linewidth} L{0.30\linewidth}@{}}
\caption{Reductions to scalar: $\mathrm{V}\to\mathrm{S}$ and $\mathrm{V}\times\mathrm{V}\to\mathrm{S}$}\label{tab:v_to_s}\\
\toprule
\textbf{Mnemonic} & \textbf{subop} & \textbf{Operands} & \textbf{Effect} & \textbf{Notes} \\
\midrule
\endfirsthead
\toprule
\textbf{Mnemonic} & \textbf{subop} & \textbf{Operands} & \textbf{Effect} & \textbf{Notes} \\
\midrule
\endhead
\texttt{dotc}    & 0x30 & sD, vA, vB & $sD \leftarrow \sum_{i=0}^{\mathrm{VLEN}-1}\mathrm{conj}(vA[i])\cdot vB[i]$ & Reduce to scalar \texttt{sD}. \\
\texttt{dotu}    & 0x36 & sD, \={A}, \={B} & $sD \leftarrow \displaystyle\sum_{i=0}^{\mathrm{VLEN}-1}\={A}[i]\;\={B}[i]$ & Complex dot (no conjugation). \\
\texttt{iamax.v} & 0x33 & sD, vA     & sD $\leftarrow \arg\max_i |vA[i]|$                   & Index in \texttt{sD} real half; imag:=0. \\
\texttt{sum.v}   & 0x34 & sD, \={A}  & $sD \leftarrow \displaystyle\sum_{i=0}^{\mathrm{VLEN}-1}\={A}[i]$ & Complex sum; reduces to scalar. \\
\texttt{asum.v}  & 0x35 & sD, \={A}  & $sD_{\mathrm{re}} \leftarrow \displaystyle\sum_{i=0}^{\mathrm{VLEN}-1}\lvert \={A}[i]\rvert,\; sD_{\mathrm{im}} \leftarrow 0$ & Sum of magnitudes (real result). \\
\bottomrule
\end{longtable}

%=== V x S -> V  =============================================
\subsection*{Vector $\times$ Scalar $\to$ Vector (broadcast per lane)}
\begin{longtable}{@{}l l l L{0.38\linewidth} L{0.30\linewidth}@{}}
\caption{Vector–scalar broadcast ops: $\mathrm{V}\times\mathrm{S} \to \mathrm{V}$}\label{tab:vs_to_v}\\
\toprule
\textbf{Mnemonic} & \textbf{subop} & \textbf{Operands} & \textbf{Effect} & \textbf{Notes} \\
\midrule
\endfirsthead
\toprule
\textbf{Mnemonic} & \textbf{subop} & \textbf{Operands} & \textbf{Effect} & \textbf{Notes} \\
\midrule
\endhead
\texttt{cadd.vs} & 0x18 & vD, vA, sB & vD[i] $\leftarrow$ vA[i] $+$ sB
  & Broadcast add; saturating per lane. \\
\texttt{csub.vs} & 0x19 & vD, vA, sB & vD[i] $\leftarrow$ vA[i] $-$ sB
  & Broadcast sub (vector minus scalar); saturating per lane. \\
\texttt{cmul.vs} & 0x1A & vD, vA, sB & vD[i] $\leftarrow$ vA[i] $\times$ sB
  & Complex lane-wise multiply by complex scalar; widen then truncate. \\
\texttt{cdiv.vs} & 0x1B & vD, vA, sB & vD[i] $\leftarrow$ vA[i] \,$\div$\, sB
  & $(a\,\overline{b})/|b|^2$ per lane; if $|sB|{=}0$ then lane:=0. \\
\texttt{cscale.vs} & 0x1C & vD, vA, t & vD[i] $\leftarrow$ vA[i] \,$\times$\, t
  & Real scale $t$ (Q32.32) broadcast to all lanes; widen then truncate. \\
\bottomrule
\end{longtable}




%---------------- I-type -----------------
\section*{I-type — Immediate (scalars only)}
\subsection*{OPCODE: \texttt{0x02}}
\begin{longtable}{@{}l l l L{0.38\linewidth} L{0.30\linewidth}@{}}
\caption{I-type: Immediate operations (scalar complex)}\label{tab:itype}\\
\toprule
\textbf{Mnemonic} & \textbf{subop} & \textbf{Operands} & \textbf{Effect} & \textbf{Notes} \\
\midrule
\endfirsthead
\toprule
\textbf{Mnemonic} & \textbf{subop} & \textbf{Operands} & \textbf{Effect} & \textbf{Notes} \\
\midrule
\endhead
\texttt{cloadi}   & 0x00 & sD, cIMM       & sD $\leftarrow$ cIMM
  & cIMM packed in \texttt{imm\_90} (Re/Im per spec). \\

\texttt{cadd\_i}  & 0x01 & sD, sA, cIMM   & sD $\leftarrow$ sA $+$ cIMM
  & Saturating per scalar; Q32.32 truncation as needed. \\

\texttt{cmul\_i}  & 0x02 & sD, sA, cIMM   & sD $\leftarrow$ sA $\times$ cIMM
  & Widen internally, clamp/truncate to Q32.32. \\

\texttt{csub\_i}  & 0x03 & sD, sA, cIMM   & sD $\leftarrow$ sA $-$ cIMM
  & Saturating; truncation semantics match \texttt{csub.c}. \\

\texttt{cdiv\_i}  & 0x04 & sD, sA, cIMM   & sD $\leftarrow$ sA \,$\div$\, cIMM
  & $(sA\,\overline{\text{cIMM}})/|\text{cIMM}|^2$; if $|\text{cIMM}|{=}0$ then sD:=0. \\

\texttt{cmaxabs\_i} & 0x05 & sD, sA, cIMM & sD $\leftarrow \arg\max_{x\in\{sA,\ \text{cIMM}\}} |x|$
  & Ties pick \texttt{sA}. \\

\texttt{cminabs\_i} & 0x06 & sD, sA, cIMM & sD $\leftarrow \arg\min_{x\in\{sA,\ \text{cIMM}\}} |x|$
  & Ties pick \texttt{sA}. \\

\texttt{cscale\_i} & 0x10 & sD, sA, rIMM  & sD $\leftarrow$ sA $\times$ rIMM
  & Real scale rIMM (Q32.32 in \texttt{imm\_90}); widen then truncate. \\

\bottomrule
\end{longtable}


%---------------- J-type -----------------
\section*{J-type — Conditional Jump}
\subsection*{OPCODE: \texttt{0x03}}
\begin{longtable}{@{}l l l L{0.38\linewidth} L{0.30\linewidth}@{}}
\caption{J-type: Conditional jump (single predicate)}\label{tab:jtype}\\
\toprule
\textbf{Mnemonic} & \textbf{subop} & \textbf{Operands} & \textbf{Effect} & \textbf{Notes} \\
\midrule
\endfirsthead
\toprule
\textbf{Mnemonic} & \textbf{subop} & \textbf{Operands} & \textbf{Effect} & \textbf{Notes} \\
\midrule
\endhead
\texttt{jrel} & 0x00 & offs33 &
If $s1 \neq 0$: $\mathrm{PC} \leftarrow \mathrm{PC} + \text{offs33}$ (instruction-relative). &
$s0 \equiv 0$; $s1$ is the conventional branch predicate (0/1). \\
\bottomrule
\end{longtable}

%---------------- S-type -----------------
\section*{S-type — Matrix-bank Vector Load/Store (stride implicit)}
\subsection*{OPCODE: \texttt{0x04}}
\begin{longtable}{@{}l l l L{0.38\linewidth} L{0.30\linewidth}@{}}
\caption{S-type: Matrix-bank vector load/store}\label{tab:stype}\\
\toprule
\textbf{Mnemonic} & \textbf{subop} & \textbf{Operands} & \textbf{Effect} & \textbf{Notes} \\
\midrule
\endfirsthead
\toprule
\textbf{Mnemonic} & \textbf{subop} & \textbf{Operands} & \textbf{Effect} & \textbf{Notes} \\
\midrule
\endhead
\texttt{vld} & 0x00 & vD, mbid, rc, idx16, len16 &
Load into vD the sequence:
$\text{if } rc{=}0:\ (r{=}\text{idx16},\ c{=}0..L{-}1),\;
\text{if } rc{=}1:\ (r{=}0..L{-}1,\ c{=}\text{idx16})$,
where $L = \text{len16}$ if nonzero, else $L=\mathrm{VLEN}$. &
Row stride $=1$;\; column stride $=\mathrm{VLEN}$. Elements are 128-bit complex (Re then Im). \\
\addlinespace[0.25em]
\texttt{vst} & 0x01 & vS, mbid, rc, idx16, len16 &
Store from vS to the same address pattern as \texttt{vld}. &
Vectors cover all lanes when $L{=}\mathrm{VLEN}$. \\
\addlinespace[0.25em]
\texttt{sld.xy} & 0x02 & sD, mbid, x16, y16 &
Load the single element at coordinates $(r{=}\text{y16},\, c{=}\text{x16})$ from matrix-bank \texttt{mbid} into scalar register \texttt{sD}. &
Coordinates are 0-based unsigned 16-bit. Element size is one 128-bit complex (Re then Im). Out-of-bounds coordinates trap. \\
\addlinespace[0.25em]
\texttt{sst.xy} & 0x03 & sS, mbid, x16, y16 &
Store scalar \texttt{sS} to the element at $(r{=}\text{y16},\, c{=}\text{x16})$ in matrix-bank \texttt{mbid}. &
Same addressing and trapping rules as \texttt{sld.xy}. Element is 128-bit complex (Re then Im). \\
\bottomrule
\end{longtable}







\end{document}
